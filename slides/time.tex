%% This file contains ?? slides
\begin{slide}
Caracterizando a rede de comunica��o
\begin{itemize} 
 \item A comunica��o � feita por pacotes em DS Links (multiplexa��o de canais)
 \item Usa chaves ass�ncronas
\end{itemize}
Tabela de transmiss�o de dados
\begin{center}
\begin{tabular}{|r|c|c|c|c|} \hline
 size & type 1 & type 2 & type 3 & type 4 \\ \hline \hline
 1 & 10,2 & 12,1 & 13,8 & 15,5 \\ \hline
 2 & 16,9 & 20,6 & 24,0 & 27,6\\ \hline
 3 & 23,5 & 28,9 & 34,2 & 39,5\\ \hline
 4 & 30,2 & 37,5 & 44,6 & 51,8 \\ \hline
 5 & 37,2 & 46,0 & 54,7 & 63,6 \\ \hline
 7 & 50,5 & 62,9 & 75,3 & 87,6 \\ \hline
 8 & 57,2 & 71,4 & 85,4 & 99,5 \\ \hline
 10 & 70,8 & 88,2 & 106 & 123 \\ \hline
 11 & 77,3 & 96,7 & 116 & 136 \\ \hline
 13 & 90,6 & 114 & 136 & 159 \\ \hline
 15 & 104 & 130 & 157 & 183 \\ \hline
 16 & 111 & 139 & 167 & 195 \\ \hline
\end{tabular}
\end{center}
\end{slide}

\begin{slide}
{\small
Conclus�es
\begin{itemize} 
 \item � poss�vel prever quando um escravo estar� livre (tempo de processamento). 
Calcular os tempos de transmiss�o para que os escravos sempre estejam livres. 
\begin{displaymath}
n = \frac{tempo\ de\ processamento}{tempo\ de\ trasmiss\tilde{a}o\ para\ outros}
\end{displaymath}
 \item No caso da GDU $n = \frac{200}{27} = 7,41 \Rightarrow$ \eng{speed-up} m�ximo
 \item No caso do 2\eiro n�vel $ n = 2 $ satisfaz taxa m�xima
\end{itemize}}
\end{slide}

\begin{note}
Aqui entra o \eng{slide} da implementa��o, que est� feito em camadas, no 
slide~\pageref{slide:implementa}.
\end{note}

\begin{slide}
Conclus�es finais
\begin{itemize} 
 \item A rede de decis�es globais tem \eng{speed-up} m�ximo de 7,4
 \item � capaz de processar uma RoI em 27$\mu$s = 7,4KHz (evento)
 \item Isto n�o satisfaz o ambiente do segundo n�vel, por�m � o m�ximo a ser 
atingido com esta tecnologia
 \item O \eng{speed-up} m�ximo para o 2\eiro n�vel � 3,4
 \item Isto n�o satisfaz o funcionamento para este tipo de processamento, mas tamb�m
 representa o m�ximo desta tecnologia
\end{itemize}
\end{slide}
