\chapter{Exemplo de arquivo \raw{Makefile}}
\label{ap:makefile}

Aqui apresentaremos o exemplo comentado de um arquivo tipo \raw{makefile}. Estes 
arquivos s�o utilizados juntos com o programa \raw{make.exe} para facilitar o 
processo de compila��o de programas em geral. Aqui o direcionamos para que compile 
aplica��es em InMOS C(C Concorrente para o sistema TN310). A aplica��o em quest�o 
utiliza 8 tipos diferentes de c�digos em C. Isto, em maneira nenhuma, limita o 
n�mero de processos a serem criados pois alguns deles podem usar o mesmo c�digo como
vimos no ap�ndice~\ref{ap:cfs}.

Definimos inicialmente algumas constantes como o compilador \raw{ICC} e \eng{flags} 
em \raw{ICCOPT}.
\begin{verbatim}
ICC=icc
ILINK=ilink
HCCONF=inconf
ICOLLECT=icollect
INIF=inif
IMAP=imap
ILINKOPT=-T9000 -GAMMAE -MO $(<:%.tco=%.dku)
HCCONFOPT=-GA
ICOLLECTOPT=-p level2.map
INIFOPT=
IMAPOPT=
ICCOPT=-GAMMAE -T9000 -G 
SOURCES = master.c time_cal.c time_trt.c time_pre.c time_muo.c ln0.c ln1.c global.c
OBJECTS= $(SOURCES:%.c=%.tco)

.SUFFIXES : .c .tco
\end{verbatim}

Agora define-se as diretivas de compila��o, a diretiva \raw{all} � a \eng{default} 
sendo executada se o programa \raw{make.exe} for chamado sem par�metros. Um processo
 interessante aqui � que as �nicas diretivas realmente independentes s�o as de 
compila��o dos arquivos em C, todas as outras dependem da execu��o de outras 
diretivas(indicado ao lado do sinal ``:''). Assim, se instruirmos o aplicativo para 
gerar \raw{all}(ou \raw{level2.btl}) teremos uma compila��o recursiva que passar� 
por todas as fases anteriores e necess�rias a execu��o desta fase, i.e., haver� 
compila��o do c�digo em C, ele ser� ``linkado'', configurado e a� sim ser� coletado 
num arquivo com extens�o \raw{btl}.
\begin{verbatim}
all :  level2.btl

.c.tco :
	$(ICC) $< $(ICCOPT) -o $@ -p $(<:%.c=%.mco)

level2.btl : level2.cfb
	$(ICOLLECT) level2.cfb $(ICOLLECTOPT)

level2.cfb: level2.cfs master.lku time_cal.lku time_trt.lku time_pre.lku 
            time_muo.lku ln0.lku ln1.lku global.lku
	$(HCCONF) level2.cfs $(HCCONFOPT)

master.lku : master.tco
	$(ILINK) master.tco -o master.lku -f cdebug.lnk $(ILINKOPT)

time_cal.lku : time_cal.tco
	$(ILINK) time_cal.tco -o time_cal.lku -f cdebugrd.lnk $(ILINKOPT)

time_trt.lku : time_trt.tco
	$(ILINK) time_trt.tco -o time_trt.lku -f cdebugrd.lnk $(ILINKOPT)

time_pre.lku : time_pre.tco
	$(ILINK) time_pre.tco -o time_pre.lku -f cdebugrd.lnk $(ILINKOPT)

time_muo.lku : time_muo.tco
	$(ILINK) time_muo.tco -o time_muo.lku -f cdebugrd.lnk $(ILINKOPT)

ln0.lku : ln0.tco
	$(ILINK) ln0.tco -o ln0.lku -f cdebugrd.lnk $(ILINKOPT)

ln1.lku : ln1.tco
	$(ILINK) ln1.tco -o ln1.lku -f cdebugrd.lnk $(ILINKOPT)

global.lku : global.tco
	$(ILINK) global.tco -o global.lku -f cdebugrd.lnk $(ILINKOPT)

map : level2.btl
	$(IMAP) level2.map -o level2.bcm  $(IMAPOPT)

clean:
	 del  *.tco
	 del  *.lku
	 del  *.cfb
	 del  *.btl
	 del  *.map
	 del  *.bcm
	 del  *.mco
	 del  *.dku
	 del  *.bak
\end{verbatim}
A diretiva \raw{clean} comanda o apagamento de todos os arquivos gerados durante a 
compila��o.


